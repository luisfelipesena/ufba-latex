\documentclass[12pt, a4paper]{report}
\usepackage[top=3cm,left=3cm,right=2cm,bottom=2cm]{geometry}
\linespread{1.3}
\setlength{\parindent}{1.25cm}
\usepackage{indentfirst}
\usepackage[utf8]{inputenc}
\usepackage[brazil]{babel}
\usepackage{amsmath}
\usepackage{amsthm}
\usepackage{amsfonts}
\usepackage{amssymb}
\usepackage{graphicx}
\usepackage{color}
\usepackage{multicol}
\usepackage[normalem]{ulem}
\usepackage{wrapfig}
\usepackage{caption}
\usepackage{fancybox}
\usepackage[pdfstartview=FitH]{hyperref}
\usepackage{subfigure}
\usepackage{algorithm}
\usepackage{algpseudocode}
\usepackage{float}
\usepackage{multirow}
\usepackage{siunitx}
\usepackage{rotating}
\usepackage{booktabs}
\usepackage[table]{xcolor}
\usepackage{array}
\usepackage{colortbl}

\graphicspath{{Figuras/}}

\renewcommand{\theenumii}{\alph{enumii}}
\DeclareMathOperator{\sen}{sen}
\DeclareMathOperator{\tg}{tg}
\DeclareMathOperator{\arctg}{arctg}
\DeclareMathOperator{\cotg}{cotg}
\DeclareMathOperator{\agm}{agm}

\newtheorem{thm}{Teorema}[section]
\newtheorem{dfn}{Definição}[section]
\newtheorem{prob}{Problema}[section]
\newtheorem{cor}{Corolário}[section]
\newtheorem{prop}{Proposição}[section]
\newtheorem{lem}{Lema} [section]

\newcounter{contar}

% Variáveis do projeto
\newcommand{\nomeUniversidade}{Universidade Federal da Bahia}
\newcommand{\nomeInstituto}{Instituto de Computação}
\newcommand{\nomeCurso}{Interação Humano-Computador}
\newcommand{\nomeProfessor}{Prof. Igor Sobral}
\newcommand{\nomeGrupo}{\sc{\large{Antoniel Magalhães (antoniels@ufba.br)}} \\
\sc{\large{André Costa (andre.lino@ufba.br)}} \\
\sc{\large{João Leahy (joao.leahy.ufba.br)}} \\
\sc{\large{Luis Felipe (luis.sena@ufba.br)}} \\
\sc{\large{Koichi Filho (koichifilho@ufba.br)}}
}
\newcommand{\titulo}{\sc{\Large{Projeto Prático Final: Sistema de Gestão de Monitorias (SIGA-M)}}}

\begin{document}

% Capa
\pagestyle{empty}
\begin{center}
\includegraphics[height=2.5cm]{UFBA.jpg}
\hspace{2cm}
\end{center}

\begin{center}
\sc{\large{\nomeUniversidade}} \\
\sc{\large{\nomeInstituto}} \\
\sc{\small{\nomeCurso}} \\

\vspace{4cm}

\titulo

\vspace{4.5cm}

\nomeGrupo

\vspace{5.5cm}

\textbf{Salvador - Bahia} \\
13 de julho de 2025
\end{center}

% Folha de rosto
\newpage
\begin{center}
\titulo

\vspace{4cm}

\nomeGrupo
\end{center}

\vspace{4cm}

\begin{flushright}
\begin{minipage}{8.6cm}
Projeto prático final apresentado ao professor \nomeProfessor\ 
como método avaliativo da disciplina \nomeCurso.
\end{minipage}
\end{flushright}
 
\vspace{8cm}

\begin{center}
\textbf{Salvador - Bahia} \\
13 de julho de 2025
\end{center}

% Índice
\newpage
\tableofcontents
\thispagestyle{empty}
\newpage
\setcounter{page}{1}
\pagestyle{plain}

\chapter{Etapa 1: Análise (Situação atual)}
\label{ch:analise}

\section{Tema e Objetivo Geral do Projeto}

O presente projeto visa realizar o design de IHC para o Sistema de Gestão de Monitorias (SIGA-M), uma plataforma computacional interativa destinada a otimizar os processos administrativos do programa de monitoria do Instituto de Computação (IC) da UFBA.

O objetivo geral é projetar uma solução que automatize e centralize as tarefas envolvidas no ciclo de vida da monitoria — desde a submissão dos projetos pelos professores até o cadastro final dos monitores selecionados. O foco do trabalho está no design da interface e da interação, buscando reduzir a carga de trabalho manual, minimizar erros e melhorar a comunicação entre os envolvidos (comissão de monitoria, professores e alunos).

\section{Identificação das Necessidades e Requisitos de IHC}

\subsection{Levantamento das Necessidades dos Estudantes}

O levantamento de necessidades foi realizado a partir de uma entrevista com um membro da comissão de monitoria, que atua como principal stakeholder do processo. As dores e necessidades identificadas são:

\begin{itemize}
    \item \textbf{Processo Manual e Repetitivo}: A criação de projetos de monitoria a cada semestre é manual. Professores precisam preencher documentos com informações que raramente mudam, copiando dados de editais anteriores.
    
    \item \textbf{Dificuldade de Gestão e Acompanhamento}: A comissão de monitoria enfrenta dificuldades para rastrear quais professores já enviaram seus projetos assinados, dependendo de planilhas e comunicação via e-mail.
    
    \item \textbf{Comunicação Fragmentada}: O envio de lembretes e a cobrança de pendências são feitos manualmente, consumindo tempo e energia da comissão.
    
    \item \textbf{Consolidação de Dados}: A geração da planilha final com todos os projetos para a Pró-Reitoria de Graduação (PROGRAD) é um trabalho manual de consolidação de múltiplos arquivos e links.
    
    \item \textbf{Fluxo de Trabalho Descentralizado}: O processo atual envolve múltiplos arquivos (DOCX, PDF), planilhas, e-mails e pastas em drives compartilhados, aumentando o risco de erros e perda de informação.
\end{itemize}

\subsection{Definição dos Requisitos de IHC}

Com base nas necessidades, foram definidos os seguintes requisitos para o SIGA-M:

\subsubsection{Requisitos Funcionais}

\begin{itemize}
    \item O sistema deve possuir três perfis de usuário: Administrador (comissão de monitoria), Professor e Aluno.
    
    \item Deve ser possível para o Administrador iniciar um novo ciclo de monitoria, importando a planilha de planejamento do semestre.
    
    \item O sistema deve gerar automaticamente os documentos de projeto de monitoria para cada disciplina, pré-preenchendo dados do professor, da disciplina e de um histórico de projetos.
    
    \item O Professor deve poder revisar, editar (se necessário), baixar o projeto para assinatura e fazer o upload do documento assinado.
    
    \item O Administrador deve ter um painel para visualizar o status de submissão de todos os projetos em tempo real.
    
    \item O Administrador deve poder enviar e-mails de lembrete (individualmente ou em massa) para professores com pendências.
    
    \item O sistema deve gerar a planilha final de projetos (para a PROGRAD) no formato exigido, com links para os documentos assinados.
\end{itemize}

\subsubsection{Requisitos Não-Funcionais}

\begin{itemize}
    \item \textbf{Usabilidade}: A interface deve ser clara e direta, permitindo que professores com diferentes níveis de familiaridade com tecnologia completem suas tarefas de forma rápida e eficiente.
    
    \item \textbf{Confiabilidade}: O sistema deve ser estável e garantir a integridade dos dados e documentos submetidos.
    
    \item \textbf{Segurança}: Dados sensíveis de professores e alunos (CPF, dados bancários, etc.) devem ser armazenados e transmitidos de forma segura.
\end{itemize}

\subsection{Matriz de Requisitos por Perfil de Usuário}

A tabela a seguir apresenta uma visão consolidada dos requisitos funcionais organizados por perfil de usuário, facilitando a compreensão das responsabilidades e permissões de cada tipo de usuário no sistema.

\begin{table}[H]
\centering
\caption{Comparação de Requisitos Funcionais por Perfil de Usuário}
\label{tab:requisitos-perfil}
\renewcommand{\arraystretch}{1.2}
\begin{tabular}{|p{4cm}|c|c|c|c|}
\hline
\textbf{Funcionalidade} & \textbf{Professor} & \textbf{Administrador} & \textbf{Estudante} & \textbf{Prioridade} \\
\hline
Login/Autenticação & \cellcolor{green!30}Sim & \cellcolor{green!30}Sim & \cellcolor{green!30}Sim & Alta \\
\hline
Visualizar projetos & \cellcolor{green!30}Sim & \cellcolor{green!30}Sim & \cellcolor{green!30}Sim & Alta \\
\hline
Criar projetos & \cellcolor{red!30}Não & \cellcolor{green!30}Sim & \cellcolor{red!30}Não & Alta \\
\hline
Submeter projetos & \cellcolor{green!30}Sim & \cellcolor{red!30}Não & \cellcolor{red!30}Não & Alta \\
\hline
Enviar lembretes & \cellcolor{red!30}Não & \cellcolor{green!30}Sim & \cellcolor{red!30}Não & Média \\
\hline
Candidatar-se & \cellcolor{red!30}Não & \cellcolor{red!30}Não & \cellcolor{green!30}Sim & Alta \\
\hline
Gerar relatórios & \cellcolor{red!30}Não & \cellcolor{green!30}Sim & \cellcolor{red!30}Não & Média \\
\hline
\end{tabular}
\end{table}

\section{Organização do Espaço do Problema}

\subsection{Personas}

Foram criadas duas personas para representar os perfis de usuários centrais no primeiro módulo do sistema:

\subsubsection{Persona 1: Prof. Roberto (Coordenador da Monitoria)}

\begin{center}
\fcolorbox{blue!50}{blue!10}{
\begin{minipage}{0.8\textwidth}
\vspace{0.5cm}
\begin{center}
{\Large \textbf{Prof. Roberto (45 anos)}}
\end{center}
\vspace{0.3cm}

\noindent \textbf{Objetivo:} Automatizar gestão de monitoria

\noindent \textbf{Frustração:} Processo manual repetitivo

\noindent \textbf{Nível Tech:} Alto

\noindent \textbf{Disponibilidade:} Limitada

\noindent \textbf{Motivação:} Eficiência e mais bolsas para alunos

\vspace{0.3cm}
\end{minipage}
}
\end{center}

\vspace{0.5cm}

\begin{itemize}
    \item \textbf{Ocupação}: Professor do Instituto de Computação e membro da comissão de monitoria.
    \item \textbf{Perfil}: Roberto é proativo e tecnologicamente competente. Ele dedica um tempo considerável à gestão do programa de monitoria e busca ativamente formas de otimizar o processo, pois sabe que um maior número de projetos submetidos resulta em mais bolsas para os alunos do instituto.
    \item \textbf{Citação}: "Todo semestre é a mesma correria. Tenho que ficar conferindo planilhas, mandando e-mails de cobrança e torcendo para não ter esquecido de ninguém. É um trabalho manual que poderia ser facilmente automatizado."
\end{itemize}

\subsubsection{Persona 2: Profa. Ana (Professora Responsável por Disciplina)}

\begin{center}
\fcolorbox{purple!50}{purple!10}{
\begin{minipage}{0.8\textwidth}
\vspace{0.5cm}
\begin{center}
{\Large \textbf{Profa. Ana (38 anos)}}
\end{center}
\vspace{0.3cm}

\noindent \textbf{Objetivo:} Submissão rápida e simples

\noindent \textbf{Frustração:} Documentos perdidos/retrabalho

\noindent \textbf{Nível Tech:} Médio

\noindent \textbf{Disponibilidade:} Muito limitada

\noindent \textbf{Motivação:} Cumprir obrigação eficientemente

\vspace{0.3cm}
\end{minipage}
}
\end{center}

\vspace{0.5cm}

\begin{itemize}
    \item \textbf{Ocupação}: Professora do Instituto de Computação.
    \item \textbf{Perfil}: Ana é uma professora dedicada, com uma agenda cheia de aulas, reuniões e orientações de pesquisa. Ela valoriza o programa de monitoria, mas vê o processo burocrático de submissão do projeto como mais uma tarefa administrativa em sua longa lista de afazeres.
    \item \textbf{Citação}: "Eu nunca lembro onde salvei o arquivo do semestre passado. Acabo tendo que preencher tudo de novo, mesmo sabendo que a descrição da monitoria de Algoritmos não muda quase nada de um ano para o outro."
\end{itemize}

\subsection{Cenários}

A seguir, apresentamos cenários de uso detalhados para cada persona, com fluxogramas que ilustram visualmente os passos e decisões envolvidos no processo.

\subsubsection{Cenário 1: Prof. Roberto iniciando o ciclo de monitoria}

\begin{figure}[H]
\centering
\includegraphics[width=0.2\textwidth]{roberto.png}
\caption{Fluxo de interação do Prof. Roberto iniciando um novo ciclo de monitoria}
\label{fig:cenario-roberto}
\end{figure}

\textbf{Descrição narrativa:}
Roberto acessa o SIGA-M e navega para a área de Administração. Ele clica em "Iniciar Novo Ciclo de Monitoria". O sistema solicita a planilha de planejamento do semestre 2025.1. Ele faz o upload do arquivo. Em poucos segundos, o sistema exibe uma mensagem: "Ciclo 2025.1 iniciado com sucesso. 62 projetos de monitoria gerados e notificações enviadas aos professores responsáveis."

Em seu painel, Roberto vê o status "Projetos Submetidos: 0/62". Uma semana depois, ele acessa novamente o painel, que agora marca "45/62". Ele filtra a lista para exibir apenas os professores com status "Pendente" e clica no botão "Enviar Lembrete para Todos os Pendentes". O sistema confirma o envio de 17 e-mails de lembrete.

\subsubsection{Cenário 2: Profa. Ana submetendo seu projeto}

\begin{figure}[H]
\centering
\includegraphics[width=0.2\textwidth]{ana.png}
\caption{Fluxo de submissão de projeto pela Profa. Ana}
\label{fig:cenario-ana}
\end{figure}

\textbf{Descrição narrativa:}
Ana recebe um e-mail com o assunto "SIGA-M: Seu projeto de monitoria para a disciplina de Algoritmos e Estrutura de Dados I está pronto". O e-mail contém um link direto. Ao clicar, ela é levada à página de login do SIGA-M. Após se autenticar, ela vê o documento de seu projeto já preenchido. Ela lê rapidamente e concorda com o conteúdo, que é o mesmo do ano anterior. Ela clica em "Baixar para Assinatura", abre o PDF, aplica sua assinatura digital e salva o arquivo. Em seguida, na mesma página, ela arrasta e solta o PDF assinado na área indicada. Uma mensagem de sucesso aparece: "Projeto enviado com sucesso!". O processo inteiro levou menos de cinco minutos.

\subsection{Análise Hierárquica de Tarefas (HTA)}

A Análise Hierárquica de Tarefas apresenta a decomposição estruturada do objetivo principal do professor no sistema SIGA-M. O diagrama a seguir ilustra visualmente a hierarquia de tarefas e subtarefas envolvidas no processo de submissão de projetos de monitoria.

\begin{figure}[H]
\centering
\includegraphics[width=0.95\textwidth]{hta.png}
\caption{Análise Hierárquica de Tarefas - Submeter projeto de monitoria}
\label{fig:hta-diagram}
\end{figure}

\textbf{Objetivo principal}: Submeter o projeto de monitoria

\textbf{Planos de ação}:
\begin{itemize}
    \item \textbf{Plano 0}: Executar tarefas 1-8 em sequência
    \item \textbf{Plano 3}: Se dados estiverem incorretos ou incompletos na tarefa 3.1 ou 3.2, executar 3.3
    \item \textbf{Plano 3.3}: Executar 3.3.1, 3.3.2 e 3.3.3 em sequência
\end{itemize}

\textbf{Condições de ativação}:
\begin{itemize}
    \item A tarefa 3.3 (edição) só é executada se o conteúdo não estiver adequado
    \item A confirmação (tarefa 7) só ocorre após upload bem-sucedido
    \item O usuário pode sair do sistema (tarefa 8) a qualquer momento após a confirmação
\end{itemize}

\chapter{Etapa 2: Síntese (Intervenção)}
\label{ch:sintese}

Esta etapa apresenta a proposta de intervenção sob a forma de um protótipo interativo do sistema SIGA-M, focando no fluxo essencial entre coordenador e professor para o ciclo de criação e aprovação de projetos de monitoria.

\section{Modelo de Interação}

O modelo de interação apresenta o fluxo simplificado de navegação entre as telas principais do sistema, concentrando-se no processo fundamental de criação, submissão e aprovação de projetos de monitoria. O desenvolvimento partiu de um protótipo de baixa fidelidade (Figura \ref{fig:excalidraw}) que estabeleceu as bases para o design final.

\begin{figure}[H]
\centering
\includegraphics[width=0.8\textwidth]{figma/excalidraw-design.png}
\caption{Protótipo de baixa fidelidade - Base inicial do design}
\label{fig:excalidraw}
\end{figure}

O fluxo principal simplificado segue duas jornadas essenciais:
\begin{itemize}
    \item \textbf{Jornada do Professor}: Login → Criar/Editar Projeto → Enviar Projeto
    \item \textbf{Jornada do Administrador}: Login → Visualizar Projetos → Aprovar/Assinar → Iniciar Processo de Bolsa
\end{itemize}

\subsection{Fluxo de Interação Principal}

O modelo de interação foi otimizado para os dois perfis principais:

\subsubsection{Fluxo do Professor}
\begin{enumerate}
    \item \textbf{Login}: Autenticação via credenciais institucionais
    \item \textbf{Dashboard}: Visualização de projetos existentes ou opção de criar novo
    \item \textbf{Criação de Projeto}: Formulário com dados pré-preenchidos quando disponíveis
    \item \textbf{Revisão}: Verificação das informações do projeto
    \item \textbf{Envio}: Submissão do projeto para aprovação do coordenador
    \item \textbf{Confirmação}: Feedback visual de sucesso no envio
\end{enumerate}

\subsubsection{Fluxo do Administrador (Coordenador)}
\begin{enumerate}
    \item \textbf{Login}: Acesso com perfil administrativo
    \item \textbf{Dashboard}: Visão geral de todos os projetos submetidos
    \item \textbf{Análise}: Revisão detalhada de cada projeto
    \item \textbf{Aprovação}: Assinatura digital e aprovação do projeto
    \item \textbf{Inicialização}: Ativação do processo de bolsa de monitoria
    \item \textbf{Notificação}: Sistema notifica automaticamente os envolvidos
\end{enumerate}

\subsection{Tratamento de Erros e Recuperação}

O sistema implementa mecanismos essenciais de tratamento de erros:

\begin{itemize}
    \item \textbf{Validação de Login}: Feedback claro para credenciais inválidas
    \item \textbf{Salvamento Automático}: Prevenção de perda de dados durante criação de projetos
    \item \textbf{Confirmações}: Diálogos de confirmação antes de ações críticas
    \item \textbf{Navegação Segura}: Sempre com opção de voltar sem perder progresso
\end{itemize}

\section{Link de Acesso ao Protótipo}

O protótipo interativo está disponível na plataforma Figma através do seguinte link:

\url{https://www.figma.com/proto/meTbBaQdqBHlvtzBEb9ehF/Sistema-de-Monitoria-IC?page-id=902:2983&node-id=982-10463&t=YUf33KtGKI4XtKzN-0&scaling=min-zoom&content-scaling=fixed&starting-point-node-id=982:10463&show-proto-sidebar=1}

O protótipo permite a simulação completa do fluxo simplificado professor-coordenador, possibilitando a navegação interativa pelas funcionalidades essenciais do sistema.

\section{Vídeo-Demo do Sistema}

O vídeo demonstrativo do sistema SIGA-M está disponível através do seguinte link:

\url{https://drive.google.com/file/d/1VY4ASzuriJqoO2uJ_hX88cgilz3mBToo/view?usp=sharing}

O vídeo apresenta a explanação oral do projeto e uma demonstração completa do protótipo, mostrando os principais fluxos de interação e funcionalidades do sistema.

\section{Telas e Decisões de Projeto}

As capturas de tela do protótipo e as justificativas das decisões de design são apresentadas a seguir, focando no fluxo principal:

\subsection{Tela de Login}

\begin{figure}[H]
\centering
\includegraphics[width=0.8\textwidth]{figma/Login.png}
\caption{Tela de Login do Sistema SIGA-M}
\label{fig:login}
\end{figure}

\textbf{Decisões de Design}:
\begin{itemize}
    \item \textbf{Identidade Visual}: Utilização das cores institucionais da UFBA (azul) para criar familiaridade
    \item \textbf{Simplicidade}: Layout limpo com foco nos campos essenciais
    \item \textbf{Diferenciação de Perfis}: Login único com identificação automática do tipo de usuário
    \item \textbf{Segurança Visual}: Campo de senha com opção de visualização
\end{itemize}

\subsection{Formulário de Criação de Projeto}

\begin{figure}[H]
\centering
\includegraphics[width=0.9\textwidth]{figma/projeto-monitoria.png}
\caption{Interface de Criação de Projeto de Monitoria}
\label{fig:projeto-monitoria}
\end{figure}

\textbf{Decisões de Design}:
\begin{itemize}
    \item \textbf{Formulário Estruturado}: Campos organizados logicamente seguindo o fluxo do documento oficial
    \item \textbf{Pré-preenchimento Inteligente}: Dados do professor e histórico carregados automaticamente
    \item \textbf{Validação em Tempo Real}: Indicadores visuais de campos obrigatórios e validados
    \item \textbf{Ações Claras}: Botões de "Salvar Rascunho" e "Enviar para Aprovação" bem posicionados
    \item \textbf{Navegação Contextual}: Breadcrumb indicando posição no processo
\end{itemize}

\subsection{Dashboard do Coordenador}

\begin{figure}[H]
\centering
\includegraphics[width=0.8\textwidth]{figma/Dashboard_coodenador.png}
\caption{Dashboard do Coordenador - Visão de Projetos}
\label{fig:dashboard}
\end{figure}

\textbf{Decisões de Design}:
\begin{itemize}
    \item \textbf{Visão Consolidada}: Lista de todos os projetos com status visual claro
    \item \textbf{Filtros Rápidos}: Opções para filtrar por status (pendente, aprovado, em análise)
    \item \textbf{Ações em Lote}: Possibilidade de aprovar múltiplos projetos
    \item \textbf{Informações Essenciais}: Exibição de dados críticos (professor, disciplina, data de submissão)
    \item \textbf{Acesso Rápido}: Links diretos para visualizar detalhes de cada projeto
\end{itemize}

\subsection{Paleta de Cores}

\begin{figure}[H]
\centering
\includegraphics[width=0.8\textwidth]{figma/Cores.png}
\caption{Paleta de Cores Utilizada no Sistema}
\label{fig:cores}
\end{figure}

\textbf{Justificativa da Paleta}:
\begin{itemize}
    \item \textbf{Azul Institucional}: Cor primária da UFBA mantida para consistência
    \item \textbf{Verde de Aprovação}: Para indicar projetos aprovados e ações positivas
    \item \textbf{Amarelo de Pendência}: Para projetos aguardando análise
    \item \textbf{Cinzas Neutros}: Para elementos de interface e textos secundários
    \item \textbf{Alto Contraste}: Garantia de acessibilidade visual
\end{itemize}

\subsection{Tipografia}

\begin{figure}[H]
\centering
\includegraphics[width=0.8\textwidth]{figma/Tipografia.png}
\caption{Sistema Tipográfico do SIGA-M}
\label{fig:tipografia}
\end{figure}

\textbf{Decisões Tipográficas}:
\begin{itemize}
    \item \textbf{Hierarquia Clara}: Títulos, subtítulos e corpo de texto bem diferenciados
    \item \textbf{Legibilidade}: Fontes sans-serif para leitura em tela
    \item \textbf{Consistência}: Mesmo sistema tipográfico em todo o protótipo
    \item \textbf{Tamanhos Adequados}: Mínimo de 14px para corpo de texto
\end{itemize}

\section{Princípios de Design Aplicados}

\subsection{Foco na Tarefa Principal}

O protótipo foi desenvolvido com foco absoluto no fluxo crítico:

\begin{itemize}
    \item \textbf{Eliminação de Complexidade}: Apenas funcionalidades essenciais implementadas
    \item \textbf{Caminho Direto}: Mínimo de cliques para completar tarefas principais
    \item \textbf{Clareza de Propósito}: Cada tela tem objetivo único e claro
    \item \textbf{Feedback Imediato}: Confirmações visuais para cada ação importante
\end{itemize}

\subsection{Usabilidade Simplificada}

Considerações especiais para o contexto acadêmico:

\begin{itemize}
    \item \textbf{Familiaridade}: Interface similar a outros sistemas da UFBA
    \item \textbf{Linguagem Clara}: Terminologia alinhada com documentos oficiais
    \item \textbf{Processo Guiado}: Fluxo linear sem ramificações desnecessárias
    \item \textbf{Recuperação Fácil}: Sempre possível corrigir erros sem perder trabalho
\end{itemize}

\chapter{Etapa 3: Avaliação (Situação nova)}
\label{ch:avaliacao}

Esta etapa apresenta a avaliação do protótipo do sistema SIGA-M com potenciais usuários, verificando se os critérios de qualidade de IHC foram atendidos e se os requisitos definidos na análise foram contemplados.

\section{Método de Avaliação e Critérios}

\subsection{Método Escolhido: Teste de Usabilidade}

O teste de usabilidade foi conduzido remotamente via chamadas de vídeo, com observação direta das interações dos participantes com o protótipo no Figma. Cada sessão durou aproximadamente 20-30 minutos, incluindo execução de tarefas, think-aloud protocol e questionário pós-teste.

\subsection{Participantes}

A avaliação foi realizada com 5 participantes reais: 3 professores do Instituto de Computação (representando o perfil de usuário professor) e 2 membros de comissões semelhantes em outros departamentos (representando o perfil administrador). Todos os participantes tinha experiência prévia com processos de monitoria e níveis variados de familiaridade com ferramentas digitais.

\subsection{Critérios de Qualidade de Uso}

Os seguintes critérios de qualidade serão avaliados:

\subsubsection{Eficácia}
\begin{itemize}
    \item Taxa de sucesso na conclusão das tarefas principais
    \item Capacidade de alcançar os objetivos propostos
\end{itemize}

\subsubsection{Eficiência}
\begin{itemize}
    \item Tempo médio para completar cada tarefa
    \item Número de cliques/ações necessárias
    \item Fluidez na navegação
\end{itemize}

\subsubsection{Satisfação}
\begin{itemize}
    \item Avaliação subjetiva da experiência (escala Likert 1-5)
    \item Facilidade percebida de uso
    \item Confiança no sistema
\end{itemize}

\subsection{Instrumento de Coleta de Dados}

O instrumento incluiu:
- Roteiro de tarefas: 1) Login e criação de projeto (para professores); 2) Revisão e aprovação de projetos (para administradores).
- Questionário pós-teste: Escala Likert (1-5) para eficácia, eficiência e satisfação; perguntas abertas para feedback.
- Registro de observações: Tempo por tarefa, erros cometidos e comentários verbais.

\subsection{Resultados Quantitativos}

- Taxa de sucesso: 100% nas tarefas principais (todos completaram com sucesso).
- Tempo médio: Criação de projeto - 4.2 minutos; Aprovação - 2.8 minutos.
- Pontuações médias: Eficácia 4.8/5; Eficiência 4.6/5; Satisfação 4.7/5.
- Número médio de cliques: 8 para fluxo completo de submissão.

\subsection{Resultados Qualitativos}

- Problemas identificados: Alguns usuários sugeriram tooltips adicionais para campos complexos; um participante notou que o botão de aprovação poderia ser mais destacado.
- Comentários: "Interface intuitiva e direta" (4 participantes); "Reduziria significativamente o tempo gasto no processo atual" (todos).
- Sugestões: Adicionar notificações por email integradas; melhorar contraste em alguns elementos.
- Padrões observados: Usuários com menos experiência tech demoraram mais na revisão, mas completaram sem erros graves.

\subsection{Mudanças Implementadas}

- Adicionado tooltips explicativos em campos do formulário de projeto.
- Aumentado o tamanho e contraste do botão de aprovação no dashboard.
- Incluído feedback visual mais claro após submissão.
- Ajustado o fluxo de navegação para permitir salvamento automático de rascunhos.

\subsection{Mudanças Não Implementadas}

- Integração com email real: Não implementada devido a limitações do protótipo Figma, mas planejada para desenvolvimento futuro.
- Customização avançada de formulários: Fora do escopo inicial, para evitar complexidade desnecessária no MVP.

\chapter{Etapa 4: Relato da Experiência}
\label{ch:relato}

Esta etapa apresenta o relato da experiência de desenvolvimento do projeto, incluindo o processo de trabalho da equipe e as reflexões sobre o aprendizado em IHC.

\section{Diário de Bordo}

Este diário documenta o processo completo de desenvolvimento do projeto SIGA-M, desde a concepção inicial até a finalização, registrando as contribuições específicas de cada membro da equipe e os marcos importantes do trabalho colaborativo.

\subsection{Processo de Desenvolvimento}

O projeto foi desenvolvido ao longo de 4 semanas intensivas, com uma divisão clara de responsabilidades que permitiu aproveitar as competências específicas de cada membro da equipe.

\subsubsection{Divisão de Responsabilidades}

A equipe se organizou da seguinte forma, com todos os membros participando ativamente em múltiplas etapas:

\begin{itemize}
    \item \textbf{Antoniel Magalhães}: Foco principal na definição do tema, criação de personas e cenários, e desenvolvimento inicial do protótipo no Figma; contribuiu ativamente na análise de requisitos e revisões finais.
    \item \textbf{André Costa}: Realizou validações de usabilidade, ajustes no design do protótipo e contribuições na documentação; participou ativamente das discussões de equipe e refinamento do fluxo de interação.
    \item \textbf{João Leahy}: Foco principal na validação com usuários reais, gravação e edição do vídeo demonstrativo; contribuiu ativamente na avaliação de usabilidade e análise de feedback.
    \item \textbf{Luis Felipe}: Documentação técnica, estruturação do relatório em LaTeX, organização das entregas; foco na definição do tema e refinamento do protótipo Figma; participou ativamente em todas as etapas de síntese.
    \item \textbf{Koichi Filho}: Realizou validações de design, ajustes nos documentos e protótipo; contribuiu ativamente na HTA e na preparação da avaliação.
\end{itemize}

\subsection{Cronograma de Atividades Realizadas}

\begin{table}[H]
\centering
\begin{tabular}{|l|l|l|}
\hline
\textbf{Período} & \textbf{Atividade} & \textbf{Responsáveis} \\
\hline
9-15 Jun & Definição do tema e análise de requisitos & Todos \\
\hline
16-18 Jun & Criação de personas e cenários & Antoniel, Luis, João \\
\hline
19-25 Jun & Desenvolvimento da HTA & Koichi, André, Luis \\
\hline
26 Jun-7 Jul & Design do protótipo no Figma & Antoniel, Luis, Koichi \\
\hline
1-8 Jul & Preparação da avaliação & João, André, Koichi \\
\hline
9 Jul & Aplicação da avaliação & Todos \\
\hline
10-13 Jul & Análise dos resultados e finalização & Todos \\
\hline
\end{tabular}
\caption{Cronograma de Atividades da Equipe}
\label{tab:cronograma}
\end{table}

\subsection{Lições Aprendidas}

\subsubsection{Sobre Colaboração em Design}

\begin{itemize}
    \item Importância da comunicação clara: Reuniões regulares via Discord ajudaram a alinhar expectativas e resolver dúvidas rapidamente.
    \item Valor da divisão de tarefas por competências: Permitir que cada membro focasse em suas forças acelerou o processo enquanto todos contribuíam em revisões cruzadas.
    \item Necessidade de revisões constantes: Iterações semanais no protótipo evitaram problemas maiores no final.
\end{itemize}

\subsubsection{Sobre Design de IHC}

\begin{itemize}
    \item Importância do design centrado no usuário: A criação de personas e cenários guiou todas as decisões, garantindo relevância prática.
    \item Valor dos protótipos iterativos: Começar com baixa fidelidade permitiu validações rápidas antes de investir em designs detalhados.
    \item Necessidade de validação com usuários reais: O feedback dos testes revelou melhorias que não havíamos antecipado, reforçando princípios de usabilidade.
\end{itemize}

\section{Autoavaliação}

\subsection{Avaliação da Equipe}

A equipe demonstrou:
\begin{itemize}
    \item Colaboração efetiva através de contribuições ativas de todos os membros.
    \item Cumprimento de prazos com entregas iterativas.
    \item Qualidade nas entregas, com foco em princípios de IHC.
    \item Aprendizado conjunto sobre design centrado no usuário.
\end{itemize}

\textbf{Nota da Equipe}: 8.8/10

\subsection{Avaliação Individual}

\subsubsection{Antoniel Magalhães}
\textbf{Contribuições}: Foco na definição do tema, personas, cenários e protótipo Figma; participação ativa em todas as etapas.
\textbf{Nota Individual}: 9/10

\subsubsection{André Costa}
\textbf{Contribuições}: Validações de usabilidade, ajustes no design e documentação; participação ativa em discussões e refinamentos.
\textbf{Nota Individual}: 8.5/10

\subsubsection{João Leahy}
\textbf{Contribuições}: Foco na validação com usuários e gravação do vídeo; participação ativa na avaliação e análise de feedback.
\textbf{Nota Individual}: 8/10

\subsubsection{Luis Felipe}
\textbf{Contribuições}: Documentação técnica, estruturação LaTeX, organização das entregas; foco na definição do tema e protótipo Figma; participação ativa em síntese.
\textbf{Nota Individual}: 9/10

\subsubsection{Koichi Filho}
\textbf{Contribuições}: Validações de design, ajustes nos documentos e protótipo; participação ativa na HTA e preparação da avaliação.
\textbf{Nota Individual}: 8.5/10

\end{document} 