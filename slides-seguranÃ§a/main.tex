\documentclass[aspectratio=169,xcolor=table]{beamer}
\usepackage[utf8]{inputenc}
\usepackage[T1]{fontenc}
\usepackage{lipsum, lmodern}
\usepackage{csquotes}
\usepackage{xcolor}
\usepackage[portuguese]{babel}
\usepackage{hyperref}
\usepackage{tabularx}
\usepackage{amsmath}
\usepackage{amssymb}

% ------------------------------------------------
% Tema e Configurações do Beamer
% ------------------------------------------------
\usetheme{DCC}

% Ajuste de espaçamento entre itens
\setbeamertemplate{itemize items}[circle]
\setbeamertemplate{itemize subitem}[circle]
\setlength{\itemsep}{0.8em}
\setlength{\parskip}{0.5em}

\graphicspath{{imgs/}{./imgs/}}

\author[Luis, João, Victoria, Vinicius]{%
  \textbf{Luis} \and \textbf{João} \and \textbf{Victoria} \and \textbf{Vinicius}
}
\title{Desinformação no Meio de Segurança}
\subtitle{Uma Perspectiva da Ciência da Computação}
\institute{Universidade Federal da Bahia \\ Instituto de Computação}
\date{\today}

\begin{document}

%-------------------------------------------------
%  SLIDE DE TÍTULO
%-------------------------------------------------
\begin{frame}[plain,noframenumbering]
    \titlepage
\end{frame}

%-------------------------------------------------
%  SLIDE DE AGENDA
%-------------------------------------------------
\begin{frame}{Agenda}
    \begin{table}
        \begin{tabularx}{\textwidth}{|l|X|}
            \hline
            \textbf{Tempo} & \textbf{Conteúdo} \\
            \hline
            0–15 min  & \textbf{Fundamentos técnicos} (ecossistema, arquitetura de disseminação, modelos de ameaça, tecnologias de detecção) \\
            \hline
            15–30 min & \textbf{Estudo de caso:} Análise de Benevenuto \& Melo (CACM 2024) sobre campanhas de desinformação via WhatsApp/Telegram \\
            \hline
        \end{tabularx}
    \end{table}
\end{frame}

\setlength{\parskip}{1em}

%=================================================
\section{Fundamentos Técnicos}
%=================================================
\begin{frame}{Ecossistema \& Vetores de Desinformação}
    \begin{itemize}
        \item \textbf{Ecossistema de Desinformação:} Rede interconectada de atores, plataformas e tecnologias
        \item \textbf{Vetores Primários:} Redes sociais, aplicativos de mensagem, sites de notícias falsas
        \item \textbf{Vetores Secundários:} Influenciadores, bots automatizados, redes de amplificação
        \item \textbf{Superfície de Ataque:} Múltiplas plataformas com diferentes modelos de moderação
        \item \textbf{Cadeia de Suprimentos:} Criação → Amplificação → Disseminação → Persistência
    \end{itemize}
\end{frame}

\begin{frame}{Impactos na Segurança Cibernética}
    \begin{itemize}
        \item \textbf{Ataques de Engenharia Social:} Desinformação como vetor para phishing e malware
        \item \textbf{Comprometimento de Infraestrutura Crítica:} Manipulação de opinião pública sobre sistemas essenciais
        \item \textbf{Operações de Inteligência:} HUMINT e SIGINT comprometidas por narrativas falsas
        \item \textbf{Ataques à Cadeia de Suprimentos:} Desinformação sobre fornecedores e parceiros
        \item \textbf{Degradação da Confiança:} Erosão da credibilidade em sistemas de segurança
    \end{itemize}
\end{frame}

\begin{frame}{Arquitetura de Disseminação}
    \begin{itemize}
        \item \textbf{Modelo Hub-and-Spoke:} Contas centrais distribuem para múltiplas contas periféricas
        \item \textbf{Redes em Cascata:} Propagação através de múltiplos níveis hierárquicos
        \item \textbf{Cross-Platform Seeding:} Sincronização coordenada entre diferentes plataformas
        \item \textbf{Amplificação Artificial:} Uso de bots e sockpuppets para simular engajamento orgânico
        \item \textbf{Timing Coordenado:} Sincronização temporal para maximizar impacto algorítmico
    \end{itemize}
\end{frame}

\begin{frame}{Modelo de Ameaça: MDM Kill Chain}
    \begin{itemize}
        \item \textbf{Reconnaissance:} Identificação de alvos, temas sensíveis e vulnerabilidades narrativas
        \item \textbf{Weaponization:} Criação de conteúdo falso, deepfakes e narrativas manipuladas
        \item \textbf{Delivery:} Distribuição via bots, influenciadores pagos e redes coordenadas
        \item \textbf{Exploitation:} Aproveitamento de vieses cognitivos e câmaras de eco
        \item \textbf{Installation:} Estabelecimento de narrativas persistentes no discurso público
        \item \textbf{Command \& Control:} Coordenação de campanhas via canais privados
        \item \textbf{Actions on Objectives:} Manipulação de comportamento, eleições ou políticas públicas
    \end{itemize}
\end{frame}

\begin{frame}{Tecnologias de Detecção \& Mitigação}
    \begin{itemize}
        \item \textbf{Natural Language Processing:} BERT, GPT e modelos transformer para classificação de texto
        \item \textbf{Computer Vision:} Detecção de deepfakes e manipulação de imagens
        \item \textbf{Network Analysis:} Identificação de comportamento coordenado inautêntico
        \item \textbf{Temporal Analysis:} Detecção de padrões de burst e sincronização anômala
        \item \textbf{Graph Neural Networks:} Análise de propagação e estruturas de rede
        \item \textbf{Ensemble Methods:} Combinação de múltiplas abordagens para robustez
    \end{itemize}
\end{frame}

\begin{frame}{Arquitetura de Inteligência de Ameaças}
    \begin{itemize}
        \item \textbf{Framework MITRE ATT\&CK} adaptado para desinformação
        \item \textbf{Modelo Diamante:} Adversário - Infraestrutura - Capacidade - Vítima
        \item \textbf{Cadeia Cibernética de Ataque:} Reconhecimento → Armamento → Entrega → Exploração → Instalação → C2 → Ações
        \item \textbf{Threat Intelligence Feeds:} Integração com OSINT e IOCs específicos
        \item \textbf{Attribution Analysis:} Correlação de TTPs e infraestrutura adversarial
    \end{itemize}
\end{frame}

\begin{frame}{Taxonomia de Táticas e Técnicas Adversariais}
    \begin{table}
        \begin{tabularx}{\textwidth}{|l|X|}
            \hline
            \textbf{Tática} & \textbf{Técnicas Específicas} \\
            \hline
            \textbf{Armamento} & Fabricação de deepfakes, manipulação de contexto, criação de narrativas falsas \\
            \hline
            \textbf{Entrega} & Spam coordenado, fazendas de bots, contas falsas, astroturfing \\
            \hline
            \textbf{Persistência} & Disseminação multiplataforma, câmaras de eco, bolhas de filtro \\
            \hline
            \textbf{Comando e Controle} & Coordenação via canais privados, amplificação de sinais, manipulação de tendências \\
            \hline
            \textbf{Evasão} & Migração entre plataformas, mutação de conteúdo, ofuscação semântica \\
            \hline
        \end{tabularx}
    \end{table}
\end{frame}

\begin{frame}{Técnicas de Detecção Aplicadas}
    \begin{itemize}
        \item \textbf{Content-Based Detection:} Análise semântica, estilometria e fact-checking automatizado
        \item \textbf{Network-Based Detection:} Identificação de clusters coordenados e padrões de propagação
        \item \textbf{Behavioral Analysis:} Detecção de bots através de padrões de atividade anômalos
        \item \textbf{Multimodal Fusion:} Combinação de texto, imagem e metadados para detecção robusta
        \item \textbf{Temporal Correlation:} Análise de sincronização e timing patterns suspeitos
        \item \textbf{Cross-Platform Tracking:} Rastreamento de narrativas entre diferentes plataformas
    \end{itemize}
\end{frame}

\begin{frame}{Padrões, Regulamentos \& Ética}
    \begin{itemize}
        \item \textbf{GDPR e LGPD:} Compliance em coleta e processamento de dados pessoais
        \item \textbf{IEEE Standards:} Padrões para sistemas de IA éticos e transparentes
        \item \textbf{ISO 27001:} Gestão de segurança da informação em sistemas de detecção
        \item \textbf{Princípios Éticos:} Fairness, accountability, transparency, explicabilidade
        \item \textbf{Content Policy:} Diretrizes de plataformas e moderação de conteúdo
        \item \textbf{Academic Ethics:} IRB approval e consentimento informado para pesquisa
    \end{itemize}
\end{frame}

\begin{frame}{Casos de Uso Globais em Desinformação}
    \begin{itemize}
        \item \textbf{Agência de Pesquisa da Internet (IRA) - Rússia:} Interferência nas eleições americanas 2016/2020
        \begin{itemize}
            \item 126 milhões de usuários alcançados no Facebook, 20 milhões no Instagram
            \item Coordenação multiplataforma (Facebook, Twitter, Instagram, YouTube)
            \item Investimento de \$1.25 milhão/mês, 3.393 anúncios pagos
            \item Técnicas: personas falsas, eventos organizados, amplificação artificial
        \end{itemize}
        \item \textbf{SolarWinds/Sunburst (APT29):} Ataques à cadeia de suprimentos + desinformação
        \begin{itemize}
            \item 18.000 organizações comprometidas globalmente (Microsoft, FireEye, Cisco)
            \item Desinformação sobre atribuição, minimização do impacto real
            \item Operação orquestrada por 18 meses sem detecção
        \end{itemize}
    \end{itemize}
\end{frame}

\begin{frame}{Casos de Uso Globais em Desinformação (cont.)}
    \begin{itemize}
        \item \textbf{Campanhas Anti-Vacina e COVID-19:} Desinformação em saúde pública
        \begin{itemize}
            \item "Dozen Disinformers": 12 indivíduos geraram 65\% da desinformação anti-vacina
            \item Movimento coordenado internacional com receita estimada em \$36 milhões
            \item Técnicas: cherry-picking de estudos, apelos emocionais, teorias conspiratórias
        \end{itemize}
        \item \textbf{Guerra na Ucrânia (2022-presente):} Operações de informação em tempo real
        \begin{itemize}
            \item Vídeos falsificados profundos de autoridades ucranianas (Zelensky)
            \item Narrativas falsas: "laboratórios biológicos", "nazistas", "genocídio"
            \item Coordenação RT, Sputnik: 7 milhões de interações/dia pré-guerra
            \item Técnicas: bots multiplataforma, influenciadores pagos, deepfakes
        \end{itemize}
    \end{itemize}
\end{frame}

%=================================================
\section{Estudo de Caso: Benevenuto \& Melo (CACM 2024)}
%=================================================
\begin{frame}{Metodologia Científica do Estudo}
    \textbf{Artigo:} "Misinformation Campaigns Through WhatsApp and Telegram in Presidential Elections in Brazil" (CACM 2024)
    \begin{itemize}
        \item \textbf{Conjunto de Dados:} 1.2M mensagens WhatsApp + 3M tweets (2018-2022)
        \item \textbf{Rotulação:} Colaboração com Lupa e Aos Fatos (verificadores certificados)
        \item \textbf{Metodologia:} Abordagem de métodos mistos combinando análise quantitativa e qualitativa
        \item \textbf{Aprovação do CEP} e considerações éticas para dados sensíveis
    \end{itemize}
\end{frame}

\begin{frame}{Arquitetura Técnica de Coleta}
    \begin{itemize}
        \item \textbf{WhatsApp:} Selenium WebDriver + engenharia reversa da API WhatsApp Web
        \item \textbf{Telegram:} API Bot Oficial + raspagem de canais públicos
        \item \textbf{Twitter:} API de Pesquisa Acadêmica v2 com limitação de taxa
        \item \textbf{Pré-processamento:} Normalização UTF-8, desduplicação via MinHash LSH
        \item \textbf{Armazenamento:} MongoDB para dados não-estruturados, PostgreSQL para metadados
    \end{itemize}
\end{frame}

\begin{frame}{Algoritmos de Detecção Implementados}
    \begin{itemize}
        \item \textbf{Baseado em BERT:} Ajuste fino do BERTimbau para classificação de português
        \item \textbf{Análise de Redes:} PageRank modificado para detecção de influenciadores
        \item \textbf{Análise Temporal:} Janela deslizante + detecção de rajadas (burst detection)
        \item \textbf{Detecção de Bots:} Features engenheiradas baseadas em comportamento temporal
        \item \textbf{Métodos de Conjunto:} Random Forest + Gradient Boosting com vote weighting
    \end{itemize}
\end{frame}

\begin{frame}{Resultados Quantitativos do Artigo}
    \begin{table}
        \begin{tabularx}{\textwidth}{|l|X|}
            \hline
            \textbf{Métrica} & \textbf{Resultado} \\
            \hline
            F1-score (BERTimbau) & 0.92 (±0.03) \\
            \hline
            Acurácia Detecção de Bots & 0.89 usando Botometer-PT \\
            \hline
            Precisão@10 (classificação) & 0.87 para conteúdo viral falso \\
            \hline
            Correlação Temporal & ρ = 0.74 entre eventos políticos e picos de desinformação \\
            \hline
        \end{tabularx}
    \end{table>
    \textbf{Significância Estatística:} Todos os resultados com p < 0.01
\end{frame}

\begin{frame}{Resposta Institucional}
    \begin{itemize}
        \item \textbf{Tribunal Superior Eleitoral (TSE):} Programa de Enfrentamento à Desinformação
        \item \textbf{Fact-Checking Organizations:} Lupa, Aos Fatos, Comprova - verificação colaborativa
        \item \textbf{Plataformas Digitais:} WhatsApp, Facebook, Twitter - políticas de moderação
        \item \textbf{Academia:} Pesquisa colaborativa com universidades e institutos
        \item \textbf{Sociedade Civil:} ONGs e organizações de monitoramento de mídia
        \item \textbf{Legislação:} Marco Civil da Internet e regulamentação de plataformas
    \end{itemize}
\end{frame}

\begin{frame}{Mapeando Teoria-Prática}
    \begin{itemize}
        \item \textbf{Gap Teoria-Prática:} Algoritmos acadêmicos vs. implementação em escala real
        \item \textbf{Desafios de Deploy:} Latência, throughput e custo computacional
        \item \textbf{Adaptação Adversarial:} Evolução contínua de táticas de evasão
        \item \textbf{Validação em Produção:} A/B testing e métricas de negócio
        \item \textbf{Human-in-the-Loop:} Integração de moderadores humanos com sistemas automatizados
        \item \textbf{Feedback Loops:} Melhoria contínua baseada em dados de produção
    \end{itemize}
\end{frame>

\begin{frame}{Lições Aprendidas \& Futuro}
    \begin{itemize}
        \item \textbf{Lições Aprendidas:}
        \begin{itemize}
            \item Necessidade de abordagem multidisciplinar (CS + Sociologia + Psicologia)
            \item Importância de datasets eticamente coletados e representativos
            \item Trade-offs entre precisão, explicabilidade e fairness
        \end{itemize}
        \item \textbf{Direções Futuras:}
        \begin{itemize}
            \item Federated Learning para preservação de privacidade
            \item Multimodal deepfake detection usando GANs adversariais
            \item Real-time detection com stream processing (Kafka + Storm)
            \item Cross-lingual e cross-cultural analysis
        \end{itemize}
    \end{itemize>
\end{frame>

%=================================================
\section{Perguntas e Comentários}
%=================================================
\begin{frame}{Perguntas e Comentários}
    \begin{center}
        \Huge Obrigado!
    \end{center>
\end{frame>

%=================================================
\section{Referências}
%=================================================
\begin{frame}{Referências Principais}
    \tiny
    \begin{itemize}
        \item Benevenuto, F. \& Melo, P. (2024). \textit{Misinformation Campaigns Through WhatsApp and Telegram in Presidential Elections in Brazil}. Communications of the ACM, 67(3), 45-53.
        \item Kumar, S. et al. (2021). \textit{Detection of COVID-19 Misinformation in Portuguese WhatsApp}. Lecture Notes in Computer Science, 12661, 234-247. Springer.
        \item DataSenado (2024). \textit{Percepção sobre Fake News e Eleições no Brasil}. Brasília: Senado Federal.
        \item TSE (2024). \textit{Programa Permanente de Enfrentamento à Desinformação: Relatório Técnico}. Brasília: Tribunal Superior Eleitoral.
        \item Zhou, X. \& Zafarani, R. (2020). \textit{A Survey of Fake News: Fundamental Theories, Detection Methods, and Opportunities}. ACM Computing Surveys, 53(5), 1-40.
    \end{itemize>
\end{frame>

\end{document}
